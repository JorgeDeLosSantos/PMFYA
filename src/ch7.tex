\chapter{Procesamiento de imágenes}

El procesamiento digital de imágenes (PDI o DIP por sus siglas en inglés) es un campo de 
investigación científica muy interesante y cuyas aplicaciones son tan diversas, tales 
como la medicina, el control de calidad en la industria, astronomía, visión artificial, etc. 
Lo anterior nos hace deducir que para abarcar "decentemente" la mayoría de los tópicos 
comunes del PDI se necesitaría más de un libro completo, por lo cual se pretende aclarar 
que en este capítulo se tratarán solamente algunos temas con un nivel de detalle elemental, 
con la esperanza de que sirva al amable lector como una breve introducción y sobre todo, 
en medida de lo posible, motivarle para adentrarse en tan maravilloso mundo.

\section{Conceptos iniciales del procesamiento de imágenes}

\subsection{¿Qué es el procesamiento de imágenes?}

El procesamiento digital de imágenes  es el conjunto de técnicas aplicadas a imágenes 
digitales con un cierto objetivo, los cuales pueden ser: mejorar la calidad de la imagen, 
detección/reconocimiento de formas o colores, realce de ciertas regiones de la imagen, etc.

\subsection{Aplicaciones del procesamiento de imágenes}
 
\section{Importar y mostrar imágenes}

En MATLAB, como en cualquier otro software de procesamiento, las imágenes son tratadas 
como matrices, cuyos elementos corresponden a un valor especifico de cada píxel que la 
compone. La función \texttt{imread} permite leer/importar imágenes en casi cualquier formato 
conocido (*.bmp, *.png, *.tiff, *.jpg, etc). La sintaxis de \texttt{imread} es muy simple, basta 
con pasar como argumento la dirección absoluta o relativa del archivo correspondiente a 
la imagen de interés, por ejemplo la siguiente instrucción lee la imagen “img.jpg” ubicada 
en la carpeta de trabajo (Current Folder).

\begin{verbatim}
	>> A=imread('img.png');
\end{verbatim}

Con el comando \texttt{whos} podemos verificar el tipo de dato con el cual MATLAB guarda la imagen importada:

\begin{verbatim}
	>> whos
	  Name        Size                Bytes  Class    Attributes
	  A         300x400x3            360000  uint8    
\end{verbatim}

MATLAB guarda la información del color de cada píxel que compone la imagen utilizando 
una matriz de tres dimensiones; las dos primeras determinan el tamaño de la imagen y la 
tercera corresponde a cada una de las capas RGB, es decir, en una matriz de mxn elementos 
se guarda la información correspondiente al color rojo, en otras similares las de color 
verde y azul. Por defecto el tipo de dato utilizado para la manipulación de imágenes 
es el uint8 (sin signo de 8 bits).\\

Para mostrar una imagen que ha sido previamente leída con imread se puede utilizar la 
función \texttt{imshow}. Véase el siguiente ejemplo:

\begin{verbatim}
	>> A=imread('imag.jpg');
	>> imshow(A);
\end{verbatim}

\begin{center}
\includegraphics[scale=0.7]{src/ch7/holland_imshow.png}
\captionof{figure}{Bla bla}
\end{center}

La función \texttt{imshow} abre una nueva ventana (figure) y muestra la imagen que ha sido 
guardada con anterioridad  (ver figura 7.1). MATLAB dispone de otras funciones como 
\texttt{image} e \texttt{imagesc}  que también muestran en pantalla las imágenes, pero que suelen 
utilizarse más para la visualización en análisis de datos.

\section{Operaciones básicas con imágenes}

\subsection{Operaciones aritméticas}

Recuerde que una imagen en MATLAB se almacena como una matriz de mxnxp dimensiones, 
luego, es posible realizar operaciones aritméticas de suma, resta y multiplicación sobre 
ella como una matriz común, tal como se ha visto en Capítulo 2.

\subsubsection{Suma de un escalar}

Si sumamos una constante k a una matriz \textbf{A}, entonces cada elemento de \textbf{A} se incrementa en k 
unidades, lo cual se traduciría en un aumento del brillo en la imagen. Véase el ejemplo siguiente:

\begin{verbatim}
	A=imread('imag.jpg');
	k=50;
	A=A+k;
	imshow(A);
\end{verbatim}

\begin{center}
\includegraphics[scale=0.7]{src/ch7/holland_original.png}
\captionof{figure}{Imagen original}
\end{center}

\begin{center}
\includegraphics[scale=0.7]{src/ch7/holland_mas50.png}
\captionof{figure}{Imagen aumentada en 50 unidades}
\end{center}

La imagen 7.2 corresponde a la original y en la 7.3 se muestra lo que resulta de aumentar 
en 50 unidades cada uno de los pixeles que componen la imagen.

\subsubsection{Resta de un escalar}

Es evidente que la resta de un escalar es muy similar en interpretación a la suma vista 
anteriormente, solo que en este caso cada elemento de la matriz disminuye en un valor 
constante. Véase el ejemplo:

\begin{verbatim}
	A=imread('img.jpg');
	k=80;
	A=A-k;
	imshow(A);
\end{verbatim}

\begin{center}
\includegraphics[scale=0.7]{src/ch7/holland_menos50.png}
\captionof{figure}{Imagen disminuida en 50 unidades}
\end{center}

\subsection{Conversión a escala de grises}

La escala de grises es una forma de representar imágenes digitales utilizando solamente variaciones 
de grises, desde negro a blanco.\\

En MATLAB se dispone de la función \texttt{rgb2gray} para convertir una imagen dada en el modelo de color RGB 
a una imagen en escala de grises.

\begin{verbatim}
X=imread('img.png');
XG=rgb2gray(X);
imshow(XG);
\end{verbatim}

\begin{center}
\includegraphics[scale=0.7]{src/ch7/holland_gris.png}
\captionof{figure}{Imagen en escala de grises}
\end{center}

\subsection{Binarización de una imagen}

\section{Segmentación de imágenes}

\subsection{Umbralización}

\subsection{Detección de bordes}

\section{Restauración y mejoramiento de imágenes}

\subsection{Removiendo ruido}