\chapter{Arreglos de celdas, estructuras y cadenas de caracteres}

\section{Arreglos de celdas}

Un \textit{cell array} \index{cell array} o arreglo de celdas es un arreglo multidimensional que puede 
contener diversos tipos de datos e incluso otro cell array. Suelen utilizarse como agrupadores de datos de diversos tipos.

\subsection{Crear un arreglo de celdas}

La manera más \textit{común} de definir un cell array es utilizando llaves como delimitadores, 
comas y/o puntos y comas como separadores (funcionando de la misma forma que en matrices). 
Por ejemplo:

\begin{verbatim}
	>> A={'Ana',rand(3),true}
	A = 
	    'Ana'    [3x3 double]    [1]
	>> whos
	  Name      Size            Bytes  Class    Attributes
	  A         1x3               415  cell     
\end{verbatim}

Puede notar que se han introducido elementos de diversos tipos y tamaños, aunque claro que funciona 
también para elementos del mismo tipo:

\begin{verbatim}
	>> A={1,2,3;0,2,1;2,3,1}
	A = 
	    [1]    [2]    [3]
	    [0]    [2]    [1]
	    [2]    [3]    [1]
\end{verbatim}

Además, puede definir un arreglo de celdas vacío utilizando la función cell, cuya sintaxis es:

\begin{verbatim}
	>> C = cell(m,n);
\end{verbatim}

Donde m y n especifican el tamaño del arreglo. Una vez creado el cell array vacío puede utilizarse 
la asignación \textit{uno a uno} para rellenar cada posición del mismo, por ejemplo:

\begin{verbatim}
	>> C=cell(2,3);
	>> C{1,1}='A';
	>> C{1,2}='X';
	>> C{1,3}=2;
	>> C{2,1}=true;
	>> C{2,2}=false;
	>> C{2,3}='T';
	>> C
	C = 
	    'A'    'X'    [2]
	    [1]    [0]    'T'
\end{verbatim}

\subsection{Indexado de un cell array}


\section{Estructuras}

Las estructuras (\texttt{struct}) son un tipo de arreglo multidimensional en MATLAB, que permiten almacenar 
datos de diversos tipos, cada dato se almacena en un campo definido mediante un nombre.

\subsection{Crear una estructura de datos}

Una estructura de datos se puede definir utilizando la sintaxis:

\begin{verbatim}
	>> NombreEstructura.NombreCampo = Valor
\end{verbatim}

Donde \textit{NombreEstructura} es el nombre de la estructura, \textit{NombreCampo} el nombre del campo, y \textit{Valor} 
el valor asignado a ese campo de la estructura, el cual puede ser una variable o dato de cualquier tipo, incluyendo 
otros tipos de arreglos o matrices. Note que el nombre de la estructura y el del campo están separados por un punto.\\





\begin{verbatim}
	>> Alumnos.nombre='Juan Pérez';
	>> Alumnos.edad=20;
	>> Alumnos.notas=[10,8,9];
	>> Alumnos
	Alumnos = 
	    nombre: 'Juan Pérez'
	      edad: 20
	     notas: [10 8 9]
	>> whos Alumnos
	  Name         Size            Bytes  Class     Attributes

	  Alumnos      1x1               580  struct      
\end{verbatim}

\subsection{Accediendo a campos de una estructura}



\section{Cadenas de caracteres}

Las cadenas de caracteres o strings (aquí usaremos indistintamente ambos términos) son un tipo 
de dato común en la mayoría de los lenguajes de programación de alto nivel, consisten en una 
serie de caracteres que representan una palabra, frase, texto o cualquier otra representación 
mediante símbolos propios de un sistema de escritura.\\

Como ya sabemos en MATLAB no hace falta declarar el tipo de cada variable, pero es necesario 
utilizar cierta sintaxis para que el intérprete reconozca cada tipo de dato, así, para crear 
una cadena de caracteres es necesario delimitar su contenido entre comillas simples, por ejemplo:

\begin{verbatim}
	>> txt='Programación en MATLAB'
	txt =
	Programación en MATLAB
	>> whos
	  Name      Size            Bytes  Class    Attributes
	  txt       1x22               44  char    
\end{verbatim}

\subsection{Concatenación}

La concatenación de cadenas de caracteres es la operación de unir dos o más cadenas en una nueva. 
Una forma de concatenar strings es utilizando la notación de corchetes, véase el ejemplo a continuación:

\begin{verbatim}
	>> cad1='Hola, ';
	>> cad2='bienvenido';
	>> cad3=' a este curso de MATLAB';
	>> cad_res=[cad1,cad2,cad3]
	cad_res =
	Hola, bienvenido a este curso de MATLAB
\end{verbatim}

Además de lo anterior MATLAB también dispone de una función \textit{especializada}  para esta tarea: 
strcat. La sintaxis es muy sencilla, a saber:

\begin{verbatim}
	>> concStr=strcat(s1, s2, s3, …, sN);
\end{verbatim}

Siendo concStr la cadena resultante, y s1, s2, s3 y sN una lista de strings separadas por comas, 
y que son los que habrán de concatenarse. Por ejemplo:

\begin{verbatim}
	>> cad=strcat('Esto es una',' cadena concatenada',' con strcat')
	cad =
	Esto es una cadena concatenada con strcat
\end{verbatim}


\begin{tcolorbox}[title=De la concatenación]
En muchos otros lenguajes de programación es común utilizar el operador de suma (+) para la concatenación de strings, 
por lo cual puede resultar \textit{tentador} intentarlo en MATLAB, pero  esto produce un resultado diferente al esperado. 
Lo que MATLAB hace es sumar elemento a elemento el valor correspondiente en código ASCII de cada una de las letras 
que componen el arreglo de caracteres, por lo cual se deduce además que es imposible operar de esta forma con cadenas 
de longitudes diferentes. Luego, si se suman dos cadenas de igual longitud, entonces MATLAB devolverá un vector de 
tipo double.
\end{tcolorbox}


\subsection{Mayúsculas y minúsculas}

Si necesita representar una cadena de caracteres solo mediante mayúsculas o minúsculas, MATLAB 
proporciona funciones para cada caso. Con upper se convierte una cadena pasada como argumento 
en su representación en mayúsculas:

\begin{verbatim}
	>> s=upper('hola mundo')
	s =
	HOLA MUNDO
\end{verbatim}

Usando lower puede hacerlo para el caso de minúsculas:

\begin{verbatim}
	>> s=lower('MATLAB es divertido')
	s =
	matlab es divertido
\end{verbatim}

\subsection{Buscar y remplazar strings}

\subsubsection{Buscar en una cadena de texto}

Actualmente es común que millones de personas busquen a diario en el internet cualquier 
diversidad de temas o palabras claves de algún interés, obteniendo respuesta en milésimas 
de segundo. Claro que los algoritmos de búsqueda que implementan los buscadores de la web 
como el omnipresente Google son muy sofisticados y basan sus resultados en conceptos muy 
definidos de clasificación y relevancia.\\

En esta sección veremos como MATLAB permite \textit{buscar} palabras, frases y caracteres 
dentro de una cadena de texto. Obviando las diferencias de complejidad y utilidad, esto es 
muy similar a lo mencionado en líneas anteriores.\\

La función más común para buscar un determinado patrón de caracteres es strfind cuya sintaxis es:

\begin{verbatim}
	>> strfind(texto, busca);
\end{verbatim}

Donde texto es la cadena de caracteres en donde se buscará el patrón definido en la 
variable busca. Y  como es \textit{costumbre} en este texto, iremos a por un ejemplo:

\begin{verbatim}
	>> texto='Anita lava la tina';
	>> busca='lava';
	>> strfind(texto,busca)
	ans =
	     7
\end{verbatim}

Y ahora, ¿por qué nos devuelve un 7?, MATLAB devuelve la posición en la cual inicia el patrón 
o cadena buscada, en este caso la palabra lava comienza en la posición 7 de texto. Un ejemplo más:

\begin{verbatim}
	>> texto='Este es otro ejemplo';
	>> busca='o';
	>> strfind(texto,busca)
	ans =
	     9    12    20
\end{verbatim}

Puede notar que en este caso MATLAB devuelve más de un resultado, lo cual indica claramente 
que el carácter buscado se encuentra más de una vez dentro del texto, y al igual que lo 
anterior estos números indican la posición inicial del carácter o cadena buscada.\\

Puede ser que la utilidad de la función strfind hasta este punto le parezca cuasi nula, démosle 
entonces una aplicación más \textit{tangible}: dada una cadena de texto y un patrón de búsqueda, 
eliminaremos cada coincidencia encontrada:

\begin{verbatim}
	>> texto='Eliminando la vocal i de esta frase';
	>> busca='i';
	>> k=strfind(texto,busca);
	>> texto(k)='' % Eliminamos coincidencias
	texto =
	Elmnando la vocal  de esta frase
\end{verbatim}

Es posible que haya notado que incluso puede remplazar cada coincidencia por otra letra y no 
necesariamente por un string vacío. Lo anterior funciona solamente para patrones de búsqueda 
cuya longitud sea unitaria, para el resto de casos debe sustituir la última línea por:

\begin{verbatim}
	>> texto(k:k+length(busca))=''
\end{verbatim}

\subsubsection{Remplazando en una cadena de texto}

Hemos visto como buscar y luego reemplazar cierto patrón buscado en una cadena de caracteres, 
mediante la indexación de cada coincidencia, pero MATLAB \textit{facilita} el trabajo y 
proporciona la función \texttt{strrep} que reemplaza las coincidencias encontradas, la sintaxis es:

\begin{verbatim}
	>> modStr=strrep(origStr,  anterior, nuevo)
\end{verbatim}

Donde origStr es la cadena de caracteres original, anterior es el valor o parte de la cadena a 
remplazar, nuevo es el valor por el cual será remplazado, y modStr la cadena resultante al 
remplazar los valores especificados. Enseguida se muestra un ejemplo, utilizando el clásico 
trabalenguas de los tigres y trigos:

\begin{verbatim}
	>> s='tres tristes tigres tragaban trigo';
	>> srep=strrep(s,'tr','tl')
	srep =
	tles tlistes tigres tlagaban tligo
\end{verbatim}

Tal como se observa, se ha remplazado toda aparición conjunta de las letras tr por tl, conllevando 
ello a una versión \textit{enrarecida} de tan magnífico ejercicio de expresión oral.\\

La función \texttt{strrep}, como se ha visto, proporciona una herramienta para reemplazar ciertos 
valores en una cadena de caracteres, pero también está limitada a que los valores pasados como 
argumentos de entrada deben ser exactos, sin permitir flexibilidad alguna al momento de definir 
los patrones de búsqueda. Desde luego que lo anterior tiene una solución práctica y muy conocida 
en el mundo de la programación: las expresiones regulares, que se estarán tratando en una sección posterior.

\subsection{Comparar cadenas de caracteres}

Comparar cadenas de caracteres resulta muy útil en casos que se necesite hacer una selección o 
toma de decisión dependiendo de una variable cuyo valor sea un string. Para tal fin se utiliza 
la función \texttt{strcmp} devuelve un valor lógico verdadero si las cadenas comparadas son iguales 
y falso en caso contrario, la sintaxis es:

\begin{verbatim}
	>> strcmp(str1, str2);
\end{verbatim}

Donde str1 y str2 son las cadenas a comparar. Revise los siguientes ejemplos:

\begin{verbatim}
	>> strcmp('MATLAB','MATLAB')
	ans =
	     1
	>> strcmp('MATLAB','matlab')
	ans =
	     0
\end{verbatim}

Del último ejemplo puede deducir que la función strcmp es case-sensitive y aun cuando las cadenas 
sean iguales y difieran únicamente por el uso de mayúsculas o minúsculas, MATLAB devolverá un 
valor false. Para ignorar o evitar que se tome en cuenta el uso de mayúsculas o minúsculas puede 
emplear previamente una conversión a cualquiera de los casos, por ejemplo:

\begin{verbatim}
	>> strcmp(lower('MATLAB'),lower('matlab'))
	ans =
	     1
\end{verbatim}

Por alguna razón la solución anterior podría parecerle poco elegante, quizá lo mismo pensaron 
los desarrolladores de MathWorks y decidieron ofrecer una función muy similar a strcmp, con 
la diferencia de ser case-insensitive, hablamos de strcmpi: 

\begin{verbatim}
	>> strcmpi('MATLAB','matlab')
	ans =
	     1
\end{verbatim}

\subsection{Expresiones regulares}