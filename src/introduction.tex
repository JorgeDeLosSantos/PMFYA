\chapter*{Introducción}

\subsection*{Del contenido}

\textbf{Programación en MATLAB®\footnote{\emph{MATLAB® es una marca
  registrada de MathWorks, Inc.}}, fundamentos y aplicaciones} consta,
por ahora, de diez capítulos en los cuales se abordan los temas básicos
de la programación en MATLAB y algunas de sus aplicaciones. Los
capítulos son:

\begin{enumerate}
\item Fundamentos del lenguaje
\item Vectores y matrices
\item Arreglos de celdas, estructuras y cadenas de caracteres
\item Gráficas
\item Exportar e importar variables y datos
\item Matemáticas con MATLAB
\item Procesamiento de imágenes
\item Interfaces gráficas de usuario
\item Programación orientada a objetos
\item Recomendaciones generales
\end{enumerate}

El \textbf{capítulo 1} abarca los temas introductorios al lenguaje,
tales como el uso del entorno de desarrollo, los tipos de datos y
operadores, estructuras de control, ficheros de comandos, funciones, etc.
Se espera que al término del capítulo el lector sea capaza de programar 
scritps de manera estructurada y comprenda la función que desempeña 
cada instrucción escrita. \\

El \textbf{capítulo 2} proporciona información más detallada acerca de
los vectores y matrices, cuya importancia es vital para sacar el máximo
provecho del lenguaje, dado que es ahí donde reside la popularidad que
se ha ganado. Se tratan las operaciones más comunes con estructuras 
matriciales: definición/creación, operaciones aritméticas, indexación, 
rotación y redimensionado. \\

El \textbf{capítulo 3} está destinado a tratar otros tipos de datos más
avanzados, pero que de igual forma juegan un papel muy importante en el
aprendizaje. Las cadenas de caracteres se incluyeron en este capítulo
con la finalidad de hacerlas más visibles, dado que muchas veces se
tiende a subestimar en este tipo de lenguajes. \\

El \textbf{capítulo 4} trata un tema muy importante como son las
gráficas, tanto bidimensionales como tridimensionales. Se tratan los 
tipos de gráficos más comunes: líneas, barras, pastel/tarta, polígonos, 
superficies, curvas tridimensionales, esferas, planos, cilindros, y 
por supuesto, algunos temas referentes al \textit{formateo} de las gráficas, 
tales como anotaciones-textos, leyendas, etiquetas, mapas de color, estilos, 
etc.\\

El \textbf{capítulo 5} se ha orientado a las diversas formas de
interactuar con datos contenidos en formatos de aplicaciones externas,
así como el manejo de las variables creadas durante una sesión de
MATLAB, y claro, la manipulación de archivos y directorios utilizando
funciones nativas de MATLAB. \\

El \textbf{capítulo 6} aborda temas fundamentales de matemáticas, tales
como álgebra elemental, cálculo diferencial e integral, cálculo
vectorial, álgebra lineal y ecuaciones diferenciales, estos desde un
enfoque de la computación simbólica. \\

El \textbf{capítulo 7} es una introducción al procesamiento digital de
imágenes, con ejemplos muy simples de lectura/escritura de imágenes
básica, conversión entre modelos de color, operaciones aritméticas,
binarización, detección de bordes, entre otras cuestiones elementales. \\

El \textbf{capítulo 8} es una introducción al desarrollo de interfaces
gráficas de usuario en MATLAB, se abarcan los controles básicos y se
programación utilizando código puro, además de una introducción muy
somera al entorno de desarrollo integrado (GUIDE). \\

El \textbf{capítulo 9} aborda un tópico que generalmente se omite en la
mayoría de los libros de esta especie: la programación orientada a
objetos (POO). Se exponen los conceptos básicos de este paradigma de
programación, la sintaxis correspondiente al lenguaje, la organización
de ficheros de clases y el manejo de objetos. \\
 
El \textbf{capítulo 10} incluye una serie de recomendaciones que le
permitirán al programador en MATLAB el desarrollo de aplicaciones mejor
documentadas, legibles y optimizadas. \\

\subsection*{El por qué...}

Estos apuntes (o algo parecido) han nacido con la finalidad de
proporcionar una herramienta didáctica a quienes inician en el mundo de
la programación en MATLAB. Ciertamente este texto no es un libro
introductorio de programación y hay conceptos elementales que se
consideran asimilados por el lector. Sí es introductorio en lo que
respecta a sintaxis y las formas de implementar soluciones utilizando
MATLAB. Además, la idea es formar unos apuntes colaborativos, en donde
las ideas o comentarios que el lector pudiera retroalimentar sirvan para
mejorar la edición permanente que actualmente tienen estos apuntes.

\subsection*{Portada}

La imagen de portada fue obtenida de
\href{https://pixabay.com/es/nano-tecnolog\%C3\%ADa-construcci\%C3\%B3n-1480553/}{Pixabay},
el autor es Pete Linforth y está publicada bajo licencia
\href{https://creativecommons.org/publicdomain/zero/1.0/deed.es}{CC0 1.0
Universal}.


\begin{verbatim}

P.J. De Los Santos
Celaya, Guanajuato, México.
\end{verbatim}

