\chapter{Introducción a las GUIs MATLAB}

Una interfaz gráfica de usuario (GUI) es un elemento gráfico que contiene uno o más controles que están disponibles 
para interactuar con el usuario mediante un entorno visual sencillo, el cual permite la comunicación entre el usuario 
y el computador. Entre algunos de los componentes más comunes de una GUI creada en MATLAB se tienen menús, barras de 
herramientas, botones, menús desplegables, cajas de texto, entre otros.\\

En las interfaces gráficas creadas en MATLAB pueden aprovecharse todas las herramientas matemáticas y de ingeniería 
que proporciona MATLAB, permiten además la manipulación de archivos de datos, así como la interacción con otras GUIs 
y mostrar datos mediante tablas y gráficas de gran calidad.\\

Generalmente las GUIs son programadas para que respondan a la manipulación del usuario con una acción específica. 
Los controles gráficos que componen una GUI están relacionados con una rutina de programación, llamada callbacks 
en el entorno MATLAB, que se ejecuta cuando sucede  un determinado evento, que puede ser la entrada de caracteres 
mediante el teclado, el clic de un botón del mouse, o situarse sobre un objeto.

\section{Elemento figure}

En MATLAB cada interfaz gráfica está creada sobre un objeto figure, en este elemento se añaden 
todos los controles gráficos que componen la GUI. La forma más simple de definir un objeto 
figure se ejemplifica enseguida:

\begin{verbatim}
	hFig = figure;
\end{verbatim}

Donde \texttt{hFig} es el handle del elemento figure.\\

Es muy común que al momento de definir o crear un objeto figure, se especifiquen algunas 
de sus propiedades con la sintaxis siguiente:

\begin{verbatim}
	hFig = figure('Propiedad ', 'Valor ',...);
\end{verbatim}

A continuación se muestra un ejemplo característico:

\begin{verbatim}
	hFig = figure('NumberTitle','off',...
	              'MenuBar','None',...
	              'Name','Figure Ejemplo',...
	              'Position',[200 200 300 300]);
\end{verbatim}

Las propiedades más comunes de un elemento figure se muestran en la tabla siguiente:

\begin{table}[h!]
\centering
{\rowcolors{1}{}{gray!20}
\begin{tabular}{p{3cm} p{12cm}}
\rowcolor{LightBlue2} \textbf{Propiedad} & \textbf{Descripción} \\
\texttt{Color} & Establece el color de figure. El valor puede establecerse mediante 
un vector de tres elementos en formato RGB \\

\texttt{MenuBar} & Oculta o muestra la barra de menús estándar de MATLAB \\

\texttt{Name} & Título mostrado en la ventana de la figura. El valor especificado es una cadena de caracteres \\

\texttt{NumberTitle} & Determina si la numeración de los elementos figure creada automáticamente por MATLAB será visible. El valor por defecto es on, para ocultar deberá especificarse off. \\

\texttt{Position} & Especifica el tamaño de la GUI y la posición relativa a la esquina inferior
izquierda de la pantalla. El valor se establece mediante un vector de cuatro
elementos cuya estructura es la siguiente: [Distancia de la
izquierda, Distancia de la parte inferior, Ancho, Alto]; \\

\texttt{Resize} & Determina si puede modificarse el tamaño de la GUI utilizando el mouse. Los
valores aceptados son: off y on, siendo este último el valor por defecto. \\

\texttt{Toolbar} & Muestra o borra el menú de herramientas del objeto figure. \\

\texttt{Units} & Unidad de medida que se utilizará para interpretar el vector de la propiedad
position. Los valores disponibles son: centimeters, characters,
inches, normalized, point y pixels. Siendo este último el valor por defecto. \\

\texttt{Visible} & Establece si la GUI es visible. Valores: on y off. \\

\end{tabular}
\caption{Propiedades de \texttt{figure}}
\end{table}



\section{Controles gráficos (uicontrol)}

Los controles gráficos son creados mediante la función uicontrol, cuya estructura general se muestra:

\begin{verbatim}
	hCont = uicontrol('Style', 'tipo de control',...
	                  'Propiedad', 'Valor');
\end{verbatim}

La propiedad \texttt{style} define el tipo de control gráfico