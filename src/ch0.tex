\chapter*{Acerca de...}

\section*{El libro...}

\textbf{Programación en MATLAB, fundamentos y aplicaciones} forma parte del proyecto LAB DLS, cuyo objetivo es
proporcionar herramientas a los usuarios hispanohablantes de MATLAB. %, mediante un blog y un canal de YouTube.\\

El libro consta, por ahora, de 10 capítulos en los cuales se abordan los temas básicos de la programación
en MATLAB y algunas de sus aplicaciones. Los capítulos son:\\

\begin{enumerate}
\item Fundamentos del lenguaje
\item Vectores y matrices
\item Arreglos de celdas, estructuras y cadenas de caracteres
\item Gráficas
\item Exportar e importar variables y datos
\item Matemáticas con MATLAB
\item Procesamiento de imágenes
\item Interfaces gráficas de usuario
\item Programación orientada a objetos
\item Recomendaciones generales
\end{enumerate}

El capítulo 1 abarca los temas introductorios al lenguaje, tales como el uso del entorno de desarrollo, los tipos
de datos y operadores, sentencias de control, ficheros de comandos y funciones.\\

El capítulo 2 proporciona información más detallada acerca de los vectores y matrices, cuya importancia es vital 
para sacar el máximo provecho del lenguaje, dado que es ahí donde reside la popularidad que se ha ganado.\\

El capítulo 3 está destinado a tratar otros tipos de datos más avanzados, pero que de igual forma juegan un
papel muy importante en el aprendizaje. Las cadenas de caracteres se incluyeron en este capítulo con la finalidad
de hacerlas más visibles, dado que muchas veces se tiende a subestimar en este tipo de lenguajes.\\

El capítulo 4 trata un tema muy importante como son las gráficas, en dos y tres dimensiones. Se ofrecen temas
que permiten \textit{estilizar} las gráficas y cómo exportarlas en diversos formatos.\\

El capítulo 5 se ha orientado a las diversas formas de interactuar con datos contenidos en formatos de aplicaciones 
externas, así como el manejo de las variables creadas durante una sesión de MATLAB, y claro, la manipulación de
archivos y directorios utilizando funciones nativas de MATLAB.\\

Los capítulos 6 y 7 muestran algunas aplicaciones de MATLAB en la solución de problemas comunes en matemáticas 
universitarias y procesamiento digital de imágenes, respectivamente.\\

El capítulo 8 es una introducción al desarrollo de interfaces gráficas de usuario en MATLAB, y por tanto se 
abarcan las cuestiones esenciales. Si necesita una mayor referencia en este tema, puede consultar el blog del
proyecto MATLAB TYP, se está trabajando en un texto orientado a este aspecto (Título: Desarrollo de GUIs en MATLAB) 
y se espera que a finales de 2015 esté disponible una versión inicial.\\

El capítulo 9 aborda un tema que generalmente se omite en la mayoría de los libros de esta especie: la programación
orientada a objetos (POO). Se exponen los conceptos básicos de este paradigma de programación, la sintaxis correspondiente 
al lenguaje, la organización de ficheros de clases y el manejo de objetos.\\

El capítulo 10 incluye una serie de recomendaciones que le permitirán al programador en MATLAB el desarrollo de
aplicaciones mejor documentadas, legibles y optimizadas.

\section*{La licencia...}

Este texto se distribuye bajo licencia Creative Commons BY-NC-ND 2.5 MX.\\

En resumen: usted es libre de copiar, compartir y/o distribuir este contenido, siempre y cuando se 
atribuyan los créditos correspondientes al autor del mismo y que no se haga uso de este con fines
comerciales.\\

Para una descripción más detallada de la licencia puede visitar: \url{http://creativecommons.org/licenses/by-nc-nd/2.5/mx/}

% \section*{La tipografía utilizada...}

% \subsection*{Texto}

% El texto ordinario está escrito en fuente tipo [tipo] y tamaño [tam], este parrafo es un ejemplo del mismo.

% \subsection*{Código MATLAB}

% El código MATLAB ha sido resaltado con la finalidad de hacer más legible el texto, por ejemplo:

% \begin{verbatim}	
% 	[x,y]=meshgrid(0:0.1:10);
% 	z = sin(x) + cos(y);
% 	surf(x,y,z);
% 	colormap(hot);
% \end{verbatim}

% \subsection*{Tips de programación}

% Los tips de programación incluyen información que le puede ser útil al usuario al momento de escribir código
% en MATLAB, se distingue del texto ordinario mediante un recuadro como el siguiente:

% \begin{tcolorbox}[title=Tip]
% Bla bla bla bla
% \end{tcolorbox}%


\section*{El autor...}

Ingeniero Mecánico, egresado del Instituto Tecnológico de Tuxtla Gutiérrez en el Sureste de la República
Mexicana y actualmente estudiante de posgrado en el Instituto Tecnológico de Celaya. Programador en 
Python, MATLAB, Java y C/C++.\\

\textit{Pedro Jorge De Los Santos}\\
\textit{delossantosmfq@gmail.com}\\

\href{https://labdls.blogspot.mx}{\includegraphics[scale=0.1]{src/blogger_logo.png}}
\href{https://www.youtube.com/user/lab2dls}{\includegraphics[scale=0.1]{src/youtube_logo.png}}
\href{https://github.com/JorgeDeLosSantos}{\includegraphics[scale=0.08]{src/github_logo.png}}
\href{https://www.linkedin.com/in/pjdlsl}{\includegraphics[scale=0.1]{src/linkedin_logo.png}}
\href{https://plus.google.com/u/0/+pjdelossantos}{\includegraphics[scale=0.1]{src/google_logo.png}}



% Image Download from: https://pixabay.com/en/tiger-animal-relaxation-rest-look-1092497/