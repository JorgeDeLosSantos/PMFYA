\chapter{Recomendaciones generales}\label{recomendaciones-generales}

\section{Escribiendo código legible}\label{escribiendo-cuxf3digo-legible}

La legibilidad cuenta, claro que sí. Una frase de Abelson y Sussman, en
el libro Estructura e interpretación de Programas de Computadora, lo
resume de forma notable: \emph{Los programas deben escribirse para que
los lean las personas y sólo de forma circunstancial para que los
ejecuten las máquinas.} \\

De modo que cuando se escribe código, en cualquier lenguaje de
programación, es importante estructurar los programas de una forma que
resulte entendible para cualquier individuo. Entonces, se deben
desarrollar lineas de código pensando en que en algún momento alguien
más tratará de entenderlo (sí, aun cuando esto sea poco probable). \\

A continuación se listan algunas recomendaciones básicas para
estructurar un programa:

\subsection{Nombres de variables autodescriptivos}\label{nombres-de-variables-autodescriptivos}

Siempre que sea posible es preferible utilizar nombres de variables
autodescriptivos, aún cuando estos resulten un poco extensos. La
ganancia en legibilidad lo compensa todo, por ejemplo, es preferible
tener:

\begin{matlab}
fuerza = 10;
area = 20;
presion = fuerza/area;
\end{matlab}

a simplemente colocar:

\begin{matlab}
f = 10;
a = 20;
p = f/a;
\end{matlab}

\subsection{Indentación del código}\label{indentacion-del-codigo}

Una buena indentación del código permite distinguir los diversos bloques
de programación. Actualmente el editor de MATLAB proporciona
herramientas que facilitan esta tarea, simplemente seleccionando el
código y presionando Ctrl + I para un \emph{indentado inteligente}. Note
las diferencias entre los códigos siguientes:

\begin{matlab}
k = 1;
while true
if rem(k,2)==0
disp('Par');
else
disp('Impar');
end
k = k + 1;
end


k = 1;
while true
    if rem(k,2)==0
        disp('Par');
    else
        disp('Impar');
    end
    k = k + 1;
end
\end{matlab}

\subsection{Documentación del
código}\label{documentaciuxf3n-del-cuxf3digo}

\subsection{Espacios}\label{espacios}

\section{Optimizando el código}\label{optimizando-el-cuxf3digo}

\subsection{Pre-asignación
(pre-allocation)}\label{pre-asignaciuxf3n-pre-allocation}

Una de las grandes ¿ventajas? de MATLAB es su \emph{tipado dinámico}

\subsection{Vectorización}\label{vectorizaciuxf3n}
